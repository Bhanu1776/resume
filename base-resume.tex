\documentclass[10pt,a4paper,ragged2e]{altacv}
\geometry{left=2cm,right=10cm,marginparwidth=6.8cm,marginparsep=1.2cm,top=1.25cm,bottom=1.25cm}
\ifxetexorluatex
  \setmainfont{Carlito}
\else
  \usepackage[utf8]{inputenc}
  \usepackage[T1]{fontenc}
  \usepackage[default]{lato}
  \usepackage{hyperref}
  \hypersetup{
    colorlinks=true,
    linkcolor=blue,
    filecolor=magenta,      
    urlcolor=blue,
    pdftitle={Overleaf Example},
    pdfpagemode=FullScreen,
    }
\fi
\definecolor{VividPurple}{HTML}{000000}
\definecolor{SlateGrey}{HTML}{2E2E2E}
\definecolor{LightGrey}{HTML}{2E2E2E}
\colorlet{heading}{VividPurple}
\colorlet{accent}{VividPurple}
\colorlet{emphasis}{SlateGrey}
\colorlet{body}{LightGrey}
\renewcommand{\itemmarker}{{\small\textbullet}}
\renewcommand{\ratingmarker}{\faCircle}
\addbibresource{sample.bib}

\begin{document}
\name{BHANU SUNKA}
\tagline{Full Stack Developer}
% Cropped to square from https://en.wikipedia.org/wiki/Marissa_Mayer#/media/File:Marissa_Mayer_May_2014_(cropped).jpg, CC-BY 2.0
%\photo{3.3cm}{profile.jpg}
\personalinfo{%
  % Not all of these are required!
  % You can add your own with \printinfo{symbol}{detail}
  \phone{+91-9619349968}
  \email{sunkabhanu281202@gmail.com}
%   \phone{000-00-0000}
%  \mailaddress{Address, Street, 00000 County}
  \location{Mumbai - India}
%  \homepage{marissamayr.tumblr.com/}
%  \twitter{@marissamayer}
  \linkedin{\href{https://www.linkedin.com/in/bhanu1776/}{Linkedin}}
  \github{\href{https://github.com/Bhanu1776}{Github}}
  \portfolio{\href{https://bhanuportfolio.vercel.app/}{Portfolio}}
%   \orcid{orcid.org/0000-0000-0000-0000} % Obviously making this up too. If you want to use this field (and also other academicons symbols), add "academicons" option to \documentclass{altacv}
}

%% Make the header extend all the way to the right, if you want.
\begin{fullwidth}
\makecvheader
\end{fullwidth}

%% Depending on your tastes, you may want to make fonts of itemize environments slightly smaller
\AtBeginEnvironment{itemize}{\small}

%% Provide the file name containing the sidebar contents as an optional parameter to \cvsection.
%% You can always just use \marginpar{...} if you do
%% not need to align the top of the contents to any
%% \cvsection title in the "main" bar.
\cvsection[page1sidebar]{Experience}

\cvevent{\textbf{Full-Stack Developer}}{\href{https://roundtechsquare.com/}{RoundTechSquare}}{\textbf{June 2023 -- Present}}{\textbf{Remote}}
\begin{itemize}
\item Architected and optimized \textbf{frontend performance} on \textbf{client-facing platforms} using \textbf{Advanced} \textbf{React Patterns}, reducing load times by \textbf{25\%}.
\item Led \textbf{end-to-end delivery} of \textbf{3+ concurrent client projects} (incl. e-commerce \& B2B portals), achieving \textbf{95\% client satisfaction}.
\item Led \textbf{UI development} for an \textbf{e-commerce launch} and executed \textbf{UI/UX overhaul} for another client platform, improving user flow.
\item Resolved critical \textbf{post-launch bugs} (authentication, state management) under pressure for an \textbf{e-commerce platform}, ensuring \textbf{stability}.
\item Rapidly adapted to requirements, designing and implementing a full \textbf{guest checkout flow} for an \textbf{e-commerce site} overnight.
\item Proactively developed the\textbf{ 'GCP Media Viewer'} internal tool, which improved workflow. Additionally, I identified and escalated a \textbf{GCP security vulnerability}.
\item Executed robust \textbf{end-to-end system integrations} for client projects, enhancing performance and \textbf{scalability by 40\%}.

\end{itemize}

\divider

\cvevent {\textbf{React Developer Intern}}{\href{https://drive.google.com/file/d/13vIrjnTaNCF3GDbOpnyO7i07u0kJtUKV/view?usp=sharing}{Parassolutions}}{\textbf{Nov 2023 -- May 2024}}{ \textbf{Mumbai, India}}
\begin{itemize}
\item Contributed to two major client projects during the internship period. 
\smallskip
\item Developed software facilitating \textbf{banks} and \textbf{fintech clients} to manage pending loan notifications, communicate via SMS, email, and WhatsApp, improving \textbf{client interactions} and operational efficiency.
\smallskip
\item Assigned to create a portal for vendors to streamline \textbf{ID} and \textbf{business card ordering}.
\smallskip
\item Initiated the development process from scratch using \textbf{React.js} and \textbf{Typescript}, Integrated \textbf{CRUD} operations using \textbf{Redux-toolkit} to enable seamless data management.
\smallskip
\item Leveraged \textbf{Tanstack React Query} for efficient API management and data fetching.
\smallskip
\item Successfully incorporated \textbf{shadcn} styling, resulting in an aesthetically pleasing and intuitive interface for users.
\smallskip
\item Delivered a user-friendly and \textbf{efficient portal}, reducing manual processes and enhancing overall productivity for financial institutions.
\end{itemize}

%\divider

% \cvevent {\textbf{SDE Intern}}{\href{https://www.itjobxs.com/}{ITJOBXS}}{\textbf{April 2023 -- August 2023}}{ \textbf{Mumbai, India}}
% \begin{itemize}
% \item Worked on the design and development part of the \textbf{college management} web application.
% \smallskip
% \item Implemented secure login mechanisms and \textbf{role-based access control} for administrators, faculty, and students to ensure data privacy and security.
% \smallskip
% \item Achieved a powerful combination of \textbf{Next.js}, \textbf{Firebase}, and \textbf{Ant Design}, resulting in a dynamic, scalable, and visually appealing web application.
% \smallskip
% \item Enabled administrators to create, update, and manage course offerings, schedules, and syllabus within the system. Implemented features allowing faculty to upload course materials, assignments, and grades securely.
% \end{itemize}



% \divider
% %\divider
\smallskip
\smallskip
\smallskip
\smallskip
\smallskip
\smallskip
\smallskip

\cvsection{TECHNICAL SKILLS}
\smallskip
\begin{itemize}
\item \textbf{Programming Languages: }\small {JavaScript, Typescript, C++}\smallskip
  \item \textbf{Frameworks:}\small{ ReactJS, NextJs, NodeJS, ExpressJS, Redux}\smallskip
  \item \textbf{Databases: }\small{SQL, MySQL, MongoDB}
   \smallskip
   \item \textbf{Other Skills: }\small{Data Structures, Algorithms, Debugging}\smallskip
  \item \textbf{Tools: }\small{Git, Github, VS Code, Postman, Vim, Npm}\smallskip
\end{itemize}

% \cvsection{Coursework Subjects} 
% \smallskip
% \begin{itemize}
% \item \textbf{Operating System}
% \item \textbf{Computer Networks}
% \item \textbf{Object Oriented Programming}
% \item \textbf{Database Management System}
% \smallskip
% \end{itemize}


% \cvevent{Product Engineer}{Google}{23 June 1999 -- 2001}{Palo Alto, CA}

% \begin{itemize}
% \item Joined the company as employe \#20 and female employee \#1
% \item Developed targeted advertisement in order to use user's search queries and show them related ads
% \end{itemize}

%\cvsection{A Day of My Life}

% Adapted from @Jake's answer from http://tex.stackexchange.com/a/82729/226
% \wheelchart{outer radius}{inner radius}{
% comma-separated list of value/text width/color/detail}
% Some ad-hoc tweaking to adjust the labels so that they don't overlap
% \wheelchart{1.5cm}{0.5cm}{%
%   10/10em/accent!30/Sleeping \& dreaming about work,
%   25/9em/accent!60/Public resolving issues with Yahoo!\ investors,
%   5/13em/accent!10/\footnotesize\\[1ex]New York \& San Francisco Ballet Jawbone board member,
%   20/15em/accent!40/Spending time with family,
%   5/8em/accent!20/\footnotesize Business development for Yahoo!\ after the Verizon acquisition,
%   30/9em/accent/Showing Yahoo!\ employees that their work has meaning,
%   5/8em/accent!20/Baking cupcakes
% }

\clearpage

% \cvsection[page2sidebar]{Publications}

\nocite{*}

% \printbibliography[heading=pubtype,title={\printinfo{\faBook}{Books}},type=book]

% \divider

% \printbibliography[heading=pubtype,title={\printinfo{\faFileTextO}{Journal Articles}}, type=article]

% \divider

% \printbibliography[heading=pubtype,title={\printinfo{\faGroup}{Conference Proceedings}},type=inproceedings]

% %% If the NEXT page doesn't start with a \cvsection but you'd
% %% still like to add a sidebar, then use this command on THIS
% %% page to add it. The optional argument lets you pull up the
% %% sidebar a bit so that it looks aligned with the top of the
% %% main column.
% % \addnextpagesidebar[-1ex]{page3sidebar}


\end{document}
